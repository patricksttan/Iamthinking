%%%%%%%%%%%%%%%%%%%%%%%%%%%%%%%%%%%%%%%%%
% Beamer Presentation
% LaTeX Template
% Version 1.0 (10/11/12)
%
% This template has been downloaded from:
% http://www.LaTeXTemplates.com
%
% License:
% CC BY-NC-SA 3.0 (http://creativecommons.org/licenses/by-nc-sa/3.0/)
%
%%%%%%%%%%%%%%%%%%%%%%%%%%%%%%%%%%%%%%%%%

%----------------------------------------------------------------------------------------
%	PACKAGES AND THEMES
%----------------------------------------------------------------------------------------
\documentclass{beamer}

\mode<presentation> {

% The Beamer class comes with a number of default slide themes
% which change the colors and layouts of slides. Below this is a list
% of all the themes, uncomment each in turn to see what they look like.

%\usetheme{default}
%\usetheme{AnnArbor}
%\usetheme{Antibes}
%\usetheme{Bergen}
%\usetheme{Berkeley}
%\usetheme{Berlin}
%\usetheme{Boadilla}
%\usetheme{CambridgeUS}
%\usetheme{Copenhagen}
%\usetheme{Darmstadt}
%\usetheme{Dresden}
%\usetheme{Frankfurt}
%\usetheme{Goettingen}
%\usetheme{Hannover}
%\usetheme{Ilmenau}
%\usetheme{JuanLesPins}
%\usetheme{Luebeck}
\usetheme{Madrid}
%\usetheme{Malmoe}
%\usetheme{Marburg}
%\usetheme{Montpellier}
%\usetheme{PaloAlto}
%\usetheme{Pittsburgh}
%\usetheme{Rochester}
%\usetheme{Singapore}
%\usetheme{Szeged}
%\usetheme{Warsaw}

% As well as themes, the Beamer class has a number of color themes
% for any slide theme. Uncomment each of these in turn to see how it
% changes the colors of your current slide theme.

%\usecolortheme{albatross}
%\usecolortheme{beaver}
%\usecolortheme{beetle}
%\usecolortheme{crane}
%\usecolortheme{dolphin}
%\usecolortheme{dove}
%\usecolortheme{fly}
%\usecolortheme{lily}
%\usecolortheme{orchid}
%\usecolortheme{rose}
%\usecolortheme{seagull}
%\usecolortheme{seahorse}
%\usecolortheme{whale}
%\usecolortheme{wolverine}

%\setbeamertemplate{footline} % To remove the footer line in all slides uncomment this line
%\setbeamertemplate{footline}[page number] % To replace the footer line in all slides with a simple slide count uncomment this line

%\setbeamertemplate{navigation symbols}{} % To remove the navigation symbols from the bottom of all slides uncomment this line
}

\usepackage{graphicx} % Allows including images
\usepackage{booktabs} % Allows the use of \toprule, \midrule and \bottomrule in tables
\usepackage{algorithm2e}
\usepackage[noend]{algorithmic}
\usepackage{multicol}

%----------------------------------------------------------------------------------------
%	TITLE PAGE
%----------------------------------------------------------------------------------------

\title[]{Strong Reduction for the Pure $\lambda$-calculus\\by Benjamin Gr$\acute{e}$goire and Xavier Leroy} % The short title appears at the bottom of every slide, the full title is only on the title page

\author{Suzanne van den Bosch} % Your name
\date{\today} % Date, can be changed to a custom date

\begin{document}

\begin{frame}
\titlepage % Print the title page as the first slide
\end{frame}

%\begin{frame}
%\frametitle{Overview} % Table of contents slide, comment this block out to remove it
%
%  \tableofcontents
%
%\end{frame}

%----------------------------------------------------------------------------------------
%	PRESENTATION SLIDES
%----------------------------------------------------------------------------------------
%

\begin{frame}
\frametitle{$\beta$-reduction}
\begin{itemize}
\item Weak $\beta$-reduction
\item Strong $\beta$-reduction
\end{itemize}

\end{frame}

\begin{frame}
\frametitle{The Pure $\lambda$-calculus}
Terms:
\begin{equation*}
a ::= x | \lambda x.a | a_1 a_2
\end{equation*}
Rules: 
\begin{equation*}
(\lambda x.a) a'  \Rightarrow a\{x \leftarrow a'\} \;\;\;\;\;\;\;\;\;\;\;\;\;\;\;\;\;\;\;\;\;\;\;\;\;\;\;\;(\beta)
\end{equation*}
\begin{equation*}
\;\;\;\;\;\;\;\;\;\;\;\:\Gamma(a)  \Rightarrow \Gamma (a') \;\;\;\;\; if\; a \Rightarrow a'  \;\;\;\;\;\;\;\;\;\;\;\;\;\; (context)
\end{equation*}
with $\Gamma ::= \lambda x.[] \;|\; []\; a\; |\; a\; [] $.
We assume all $\lambda$-terms $a$ are strongly normalizing.
\end{frame}

\begin{frame}
\frametitle{Two computational problems}
\begin{itemize}
\item To compute the normal form $\mathcal{N}(a)$ of a closed, strongly normalizing term $a$.
\item To decide whether two closed, strongly normalizing term $a_1$ and $a_2$ are $\beta$-equivalent, written as $a_1 \approx a_2$.
\end{itemize}
\end{frame}


\begin{frame}
\frametitle{Two computational problems}
\begin{itemize}
\item \textbf{To compute the normal form $\mathcal{N}(a)$ of a closed, strongly normalizing term $a$.}
\item To decide whether two closed, strongly normalizing term $a_1$ and $a_2$ are $\beta$-equivalent, written as $a_1 \approx a_2$.
\end{itemize}
\end{frame}

\begin{frame}
\frametitle{Strong reduction by iterated symbolic weak reduction and readback}
\begin{equation*}
\mathcal{N}(a) = \mathcal{N}(\lambda x.a') = \lambda x.\mathcal{N}(a')
\end{equation*}
$\;$\\
$\;\;\;\;\;\;\;\;\;\;\;\;\;\;\;\;\;\;\;\;\;\;\;\;\;\;\;\;\;\;\;\;$Problem: $a'$ is not necessarily closed.

\end{frame}

\begin{frame}
\frametitle{The extended version}
Terms:
\begin{equation*}
b ::= x | \lambda x.b | b_1 b_2 | [\tilde{x} v_1 ... v_n]
\end{equation*}
Values:
\begin{equation*}
v ::= \lambda x.b | [\tilde{x} v_1 ... v_n]
\end{equation*}
Rules: 
\begin{equation*}
(\lambda x.b) v  \Rightarrow b\{x \leftarrow v\} \;\;\;\;\;\;\;\;\;\;\;\;\;\;\;\;\;\;\;\;\;\;\;\;\;\;\;\;(\beta_v)
\end{equation*}
\begin{equation*}
[\tilde{x} v_1 ... v_n] v  \Rightarrow [\tilde{x} v_1 ... v_n v] \;\;\;\;\;\;\;\;\;\;\;\;\;\;\;\;\;\;\;\;\;\;\;\;\;\;\;\;(\beta_s)
\end{equation*}
\begin{equation*}
\;\;\;\;\;\;\;\;\;\;\;\:\Gamma_V(a)  \Rightarrow \Gamma_v (a') \;\;\;\;\; if\; a \Rightarrow a'  \;\;\;\;\;\;\;\;\;\;\;\;\;\; (context_v)
\end{equation*}
with $\Gamma_v ::= []\; v\; |\; b\; [] $.

\end{frame}

\begin{frame}
\frametitle{Strong normalization procedure}
\begin{enumerate}
\item Normalize weakly
\item Read back 
\end{enumerate}

\end{frame}

\begin{frame}
\frametitle{Readback}
\begin{equation}
\mathcal{N}(b) = \mathcal{R}(\mathcal{V}(b))
\end{equation}
\begin{equation}
\mathcal{R}(\lambda x.b) = \lambda y.\mathcal{N}((\lambda x.b [\tilde{y}])) (y fresh)
\end{equation}
\begin{equation}
\mathcal{R}([\tilde{x} v_1 ... v_n]) = x \mathcal{R}(v_1) ... \mathcal{R}(v_n)
\end{equation}

$\mathcal{R}$ transforms values $v$ into normalized source terms $a$.
\end{frame}

\begin{frame}
\frametitle{Example}
Consider the following source term
\begin{equation*}
a = (\lambda x.x)(\lambda y.(\lambda z.z) y (\lambda t.t)).
\end{equation*}
Weak symbolic evaluation reduces $a$ to 
\begin{equation*}
v = \lambda y.(\lambda z.z) y (\lambda t.t).
\end{equation*}
The readback will restart weak symbolic evaluation on 
\begin{equation*}
b = (\lambda y.(\lambda z.z) y (\lambda t.t)) [\tilde{u}].
\end{equation*}
After the weak symbolic evaluation, the value is 
\begin{equation*}
v' = [\tilde{u} (\lambda t.t)].
\end{equation*}
Eventually, we will get
\begin{equation*}
\mathcal{N}(a) = \mathcal{R}(v) = \lambda u.u (\lambda w.w).
\end{equation*}


\end{frame}



\begin{frame}
\frametitle{Questions?}
\begin{figure}
\includegraphics[width=0.6\linewidth]{questionmark}
\end{figure}
\end{frame}
\end{document} 